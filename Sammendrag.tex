% \Section er den øverste inndelingen av kapittel for article-dokumentklassen. 
% Tittelen her med sidenummer kommer automatisk inn i innholdsfortegnelsen.
% For sammendraget brukes \section*{<navn>}. * gjør at kapittelet ikke blir
% Nummerert og ikke havner i innholdsfortegnelsen.
% Det kan legges til likvel gjennom \addcontentsline{toc}{<kapittelnivå>}{<Innholdsfortegnelsestekst>}.
\addcontentsline{toc}{section}{Sammendrag}
\section*{Sammendrag}

Sammendragstekst for oppgaven. Husk at den skal være et sammendrag av alt i rapporten, 
ikke bare hoveddelen deres.
Her vil dere merke forskjellen på enkelt linjeskift 

og dobbelt linjeskift. Vær
Obs
På
dette.

Desverre klarer jeg ikke lage et dokument som viser alt dere kommer til å komme borti av vanskeligheter.
Skulle det være noe dere sitter fast med er det bare å sende en mail, eller poste i diskusjonsforumet.

\section{Numerert kapittel}

Dette kapittelet blir automatisk lagt til i innholdsfortegnelsen. Kapittelet er bare her som eksempel.

\subsection{Nummerert underkapittel}

Igjen et eksempel som blir automatisk lagt til i innholdsfortegnelsen. 
Siste forhåndsdefinerte nivå er subsubsection.

\subsubsection{Nummert under-underkapittel}

Wee!