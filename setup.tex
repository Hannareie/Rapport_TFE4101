% Legger inn en del pakker jeg kan anbefale å bruke.
% Det blir desverre for mye arbeid å forklare alle i detalj.
% De aller fleste Latex-pakker er godt dokumentert på nett.

% Babel definerer standard autogenerert tekst for mange forskjellige språk.
% Når vi senere kommer til figurtekst og lignende, vil dere se at Babel forsikrer
% at det står Figur og ikke Figure i den genererte figurteksten. Det samme gjelder
% for innholdsfortegnelse etc.
\usepackage[norsk]{babel}

% << >> erstatter "" i referanseliste (Frivillig)
\usepackage{csquotes}

% Gjør at output PDF støtter linker.
\usepackage{hyperref}

% For mer fleksible nummererte lister.
\usepackage{enumitem}

% Definerer de aller fleste mattesymboler. Feks. Integrasjonstegn, Sum, XOR osv.
\usepackage{amsmath}

% For å inkludere bilder i rapporten. Takler ganske mange formater
\usepackage{graphicx}
% Søkepath for å finne bilder. Dette er alternativt til å skrive full path der dere legger inn bildet.
\graphicspath{{./figurer/}}
\usepackage{subcaption}

% Brukes for å få tekst til å gå over flere rader i en tabell.
\usepackage{multirow}

% Hovedpakke for tikz figurtegning
\usepackage{tikz}
% Utvidelse for 3D-tegning
\usepackage{tikz-3dplot}
\usetikzlibrary{shapes, arrows}
% Utvidelse for kretstegning. Kun denne er nødvendig for kretstegning.
\usepackage{circuitikz}

% En av pakkene som kan brukes til referanser
\usepackage[style=ieee, citestyle=numeric-comp]{biblatex}
\addbibresource{mylib.bib}