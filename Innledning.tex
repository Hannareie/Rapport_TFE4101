\section{Innledning}

Skriv generelt om hva laboppgaven handler om, at det er en del av labopplegget i TFE4101 på NTNU. 
Dere kan også skrive om rapportens struktur, hvis det er noe spesielt dere ønsker å få fram.

\subsection{Figurer}

Det finnes flust av måter å lage figurer på i latex. For å legge inn bilder er pakken graphicx brukt mest. En figur dannes innenfor et figure scope. 
Vi ser i figur~\ref{fig:inl_6cm_pika} en 6cm høy pikachu opp ned. Denne skiller seg ut fra Pikachu i figur~\ref{fig:inl_scaled_pika}~og~\ref{fig:inl_relative_pika}, som er en del av figur~\ref{fig:inl_subfigs_main}.

% OBS til avsnittet over
% ~ Brukes til å forsikre seg om at det havner på samme linje.


% \begin{figure}[<placement specifier>]
% Plassering i LaTeX kan være knotete, og er antagelig noe av det dere kommer til å bruke mest tid på.
% De vanligste er:
%   - h (plasser her i teksten)
%   - t (plasser på toppen av neste side
%   - b (plasser på bunn av denne siden)
%   - ! (Få kompilatoren til å bry seg mer om disse plasseringsegenskapene enn feks marger)
% ! fungerer ikke alene, og kommandoene er i prioritert rekkefølge fra venstre til høyre.
\begin{figure}[!htb]
    %\centering gjør at alt som følger innenfor dette scopet sentreres
    \centering
    
    % Sett inn grafikkfil \includegraphics[<formatering>]{<path/to/file.extension>}
    \includegraphics[height=6cm, angle=180]{figurer/pika.jpg}
    \caption{6cm høy Pikachu}
    \label{fig:inl_6cm_pika}
\end{figure}

\begin{figure}[!htb]
    \centering
    % Subfigure er definert i pakken subcaption.
    % bruk: \begin{subfigure}{<størrelse figurområde>}
    % \textwidth er en innebygd størrelse basert på margene på siden.
    % 0.45\textwidth betyr: Denne underfiguren tildeles 45% tilgjengelig sidebredde for tekst.
    % Dette er alltid relativt til avsatt plass. Legg spesielt merke til width=0.9\linewidth
    % i subfigur nr 2.
    \begin{subfigure}{0.45\textwidth}
        \centering
        \includegraphics[scale=0.15]{figurer/pika.jpg}
        % \caption er figurtekst. Må komme før \label
        \caption{Skalert Pikachu}
        % \label er figurreferanse til bruk for referering. Må etter \caption
        \label{fig:inl_scaled_pika}
    \end{subfigure}
    \begin{subfigure}{0.45\textwidth}
    \centering
        \includegraphics[width=0.9\linewidth]{figurer/pika.jpg}
        \caption{Størrelse er relativt til linjebredde}
        \label{fig:inl_relative_pika}
    \end{subfigure}

    % Ytterste figurtekst of referansenivå.
    \caption{Hovedfigur}
    \label{fig:inl_subfigs_main}
\end{figure}

\clearpage

\subsection{Tabeller}

For å sette inn en tabell brukes table environment Her kan jeg anbefale å bruke tabular for å lage selve tabellen. 
Tabell~\ref{tab:inl_full_tabell} viser tekstjustering, tabell~\ref{tab:inl_sparse_tabell} viser et annet linjeoppsett,
og tabell~\ref{tab:inl_multi_tabell} viser multikolonner og rader.

% \begin{table}[<placement>]
\begin{table}[!htb]
    \centering
    \caption{Vanlig tabell}
    \label{tab:inl_full_tabell}
    
    % Tabular lager selve tabellen. Defineres ved \begin{tabular}{<radformattering>}
    % | betyr vertikal linje, c, l, r er tekstjustering.
    % I tabeller skilles hvert radelement med &. 
    % Ny linje legges inn med \\
    \begin{tabular}{|c|l|r|}
        %horisontal linje \hline. 
        \hline
        \textbf{Sentrert}   & \textbf{venstrejustert}   & \textbf{høyrejustert} \\ \hline \hline
         hei                & på                        & deg                   \\ \hline 
    \end{tabular}
\end{table}

\begin{table}[!htb]
    \centering
    \caption{Tabell med mindre inndeling}
    \label{tab:inl_sparse_tabell}
    \begin{tabular}{c|c}
        \textbf{Vare}   & \textbf{Verdi} \\ \hline
        Vann            & 3kr \\
        Mer vann        & 4kr \\
        Masse vann      & 12kr \\ \hline
        Sum             & 19kr \\
    \end{tabular}
\end{table}

% Tabeller med mye formattering kan ofte bli uoversiktlige desverre. 
% Bruk tabs for å lage oversikt.
% For multirad bruk \multirow{<antall>}{<bredde>}{<tekst>}
% For multikolonne bruk \multicolumn{<antall>}{<formattering>}{<tekst>}
\begin{table}[!htb]
    \centering
    \caption{multirad og multikolonne}
    \label{tab:inl_multi_tabell}
    \begin{tabular}{|c|c|c|} 
        \hline
        \multicolumn{3}{|c|}{\textbf{Felles overskrift}} \\ \hline
        \multirow{4}{3cm}{\centering Valg}  & Ja &  \\ \cline{2-3}
                                            & Nei & \\ \cline{2-3}
                                            & Kanskje & X \\ \cline{2-3}
                                            & Vet ikke & \\ \hline
    \end{tabular}
\end{table}