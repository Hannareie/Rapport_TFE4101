\section{Resultater}

Resultatdelen er stort sett den som inneholder flest figurer og bilder. 
Denne delen av rapporten skal vise resultatene av målingene og testingen dere har gjort,
på en så oversiktlig måte som mulig. oscilloskopbilder er typisk noe man presenterer i denne delen.

I tillegg er det ofte lurt å samle lignende resultater i en tabell. Det gjør både at det er lettere
å refere til, og at det er mer oversiktlig for leser.

\subsection{Matte i LateX}

Det er en stor del som står igjen, og det er formler og formelreferering i LaTeX. I LaTeX har man forskjellige scopes,
som dere har sett i bruk flere steder. Spesielt for matte-scopet er at det er helt egne symboler, tegn og fonter definert.
For å aksessere dette benyttes enten begin equation eller \$ <matte> \$. Feks. $x+y=z$. Den samme ligningen er gjengitt i ligning~\eqref{eq:res_enkel_ligning}.
Vær obs på forskjellen mellom å referere til figur, tabell og ligning. Henholdsvis figur~\ref{fig:inl_6cm_pika}, tabell~\ref{tab:inl_multi_tabell} og ligning~\eqref{eq:res_aligned}.

\begin{equation}
    x + y = z
    \label{eq:res_enkel_ligning}
\end{equation}


\begin{equation}
    % Aligned benyttes for flere linjer med ett nummer
    \begin{aligned}
        f(x) &= 3x^2 + 4x - 3 \\
            &= 3 \cdot 4^2 + 4 \cdot 4 - 3 \\
            &= 61
    \end{aligned}
    \label{eq:res_aligned}
\end{equation}

% Align kan brukes for å nummerere hver rad. Dette er også matte-scope.
% For referering er det veldig vessentlig hvor \label{} plasseres
\begin{align}
    \label{eq:res_rad1}
    x &= 3 \\
    \label{eq:res_rad2}
    y &= 4 \\
    \label{eq:res_rad3}
    z &= x + y = 3 + 4 = 7
\end{align}

Aligned innenfor equation scope, slik som vist i ligning~\eqref{eq:res_aligned}, skiller seg fra align scope, vist i ligning~\eqref{eq:res_rad1},~\eqref{eq:res_rad2}~og~\eqref{eq:res_rad3}. 
I ligning~\eqref{eq:res_grense_integrasjon} er pakken amsmath tatt i bruk for å få spesielle symboler.

\begin{equation}
    \lim_{x \to \infty} \int_{0}^{3} 3xy \,dy 
    = \lim_{x \to \infty} \left( \frac{3}{2}x \cdot 3 - \frac{3}{2}x \cdot 0 \right)
    = \lim_{x \to \infty} \frac{9}{2}x \rightarrow \infty
    \label{eq:res_grense_integrasjon}
\end{equation}